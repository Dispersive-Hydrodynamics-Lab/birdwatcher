\documentclass[12pt]{beamer}

\input{./tex/header.tex}
\usepackage{multimedia}
%%%%%%%%%%%%%%%%%%%%%%%%%%%%%%%%%%%%%%%%%%%%%%%%%%%%%%%%%%%%%%%%%%%%%%%%%%%%%%%%
% Beginning of document items - headers, title, toc, etc...
%%%%%%%%%%%%%%%%%%%%%%%%%%%%%%%%%%%%%%%%%%%%%%%%%%%%%%%%%%%%%%%%%%%%%%%%%%%%%%%%
\title[Ouroboros]{Ouroboros}
    \author{Will Farmer\\\url{www.will-farmer.com}}
\institute{University of Colorado, Boulder\\
            Advisors: Mark Hoefer, Michelle Maiden}
\date{March 4, 2017}


%%%%%%%%%%%%%%%%%%%%%%%%%%%%%%%%%%%%%%%%%%%%%%%%%%%%%%%%%%%%%%%%%%%%%%%%%%%%%%%%
% Custom Beamer Theming
%%%%%%%%%%%%%%%%%%%%%%%%%%%%%%%%%%%%%%%%%%%%%%%%%%%%%%%%%%%%%%%%%%%%%%%%%%%%%%%%
\usetheme{Madrid}
\makeatletter

% http://tex.stackexchange.com/questions/35166/how-can-i-remove-the-institute-from-the-author-footline-on-beamer
% TLDR; Remove the Institution from default Madrid theme by redefining footline
% template.
\setbeamertemplate{footline}
{
  \leavevmode%
  \hbox{%
  \begin{beamercolorbox}[wd=.333333\paperwidth,ht=2.25ex,dp=1ex,center]{author in head/foot}%
      \usebeamerfont{author in head/foot}{\color{cugold}\insertshortauthor}%~~\beamer@ifempty{\insertshortinstitute}{}{(\insertshortinstitute)}
  \end{beamercolorbox}%
  \begin{beamercolorbox}[wd=.333333\paperwidth,ht=2.25ex,dp=1ex,center]{title in head/foot}%
    \usebeamerfont{title in head/foot}\insertshorttitle
  \end{beamercolorbox}%
  \begin{beamercolorbox}[wd=.333333\paperwidth,ht=2.25ex,dp=1ex,right]{date in head/foot}%
    \usebeamerfont{date in head/foot}\insertshortdate{}\hspace*{2em}
    \insertframenumber{} / \inserttotalframenumber\hspace*{2ex} 
  \end{beamercolorbox}}%
  \vskip0pt%
}
\makeatother

% Redefine title page to be a little more square/2d
\setbeamertemplate{title page}
{
  \begin{centering}
    \begin{beamercolorbox}[sep=8pt,center]{title}
      \usebeamerfont{title}\inserttitle\par%
      \ifx\insertsubtitle\@empty%
      \else%
        \vskip0.25em%
        {\usebeamerfont{subtitle}\usebeamercolor[fg]{subtitle}\insertsubtitle\par}%
      \fi%     
    \end{beamercolorbox}%
    \begin{beamercolorbox}[sep=8pt,center]{institute}
      \usebeamerfont{institute}\insertinstitute
    \end{beamercolorbox}
    \vskip1em\par
    \begin{beamercolorbox}[sep=8pt,center]{date}
      \usebeamerfont{date}\insertdate
    \end{beamercolorbox}%\vskip0.5em
    \begin{beamercolorbox}[sep=8pt,center]{author}
      \usebeamerfont{author}\insertauthor
    \end{beamercolorbox}
  \end{centering}
  %\vfill
}
\makeatother

\usecolortheme{wolverine}

%%%%% SET THEME
\beamertemplatenavigationsymbolsempty   % Disable navigation

%%%%% INSERT LOGO
\logo{%
    \vspace{-0.3cm}
    \makebox[0.95\paperwidth]{%
        \includegraphics[scale=0.4]{./img/nsf.png}{\color{cublack}Funded by NSF EXTREEMS-QED}
        \hfill%
        \color{cugold}CU Boulder Applied Math\includegraphics[height=0.8cm]{./img/appm.png}
    }%
}

%%%%% DEFINE COLORS
\definecolor{bgcolor}{RGB}{255,255,240}
\definecolor{cugold}{RGB}{207,184,124}
\definecolor{cublack}{RGB}{0,0,0}
\definecolor{cudarkgray}{RGB}{86,90,92}
\definecolor{culightgray}{RGB}{162,164,163}

%%%%% SET COLORS
\setbeamercolor{palette primary}{bg=cugold,fg=cublack}
\setbeamercolor{palette secondary}{bg=culightgray}
\setbeamercolor{palette tertiary}{bg=cublack}
\setbeamercolor{frametitle}{bg=cugold,fg=cublack}
\setbeamercolor{item projected}{fg=cugold,bg=black}
\setbeamercolor{itemize item}{fg=cublack,bg=black}

%%%%% List Styling
\setbeamertemplate{itemize items}[circle]
\setbeamertemplate{enumerate items}[circle]

%%%%%%%%%%%%%%%%%%%%%%%%%%%%%%%%%%%%%%%%%%%%%%%%%%%%%%%%%%%%%%%%%%%%%%%%%%%%%%%%
% Document
%%%%%%%%%%%%%%%%%%%%%%%%%%%%%%%%%%%%%%%%%%%%%%%%%%%%%%%%%%%%%%%%%%%%%%%%%%%%%%%%
\begin{document}
\frame{\titlepage}

\logo{%
    \vspace{-0.3cm}\color{cugold}CU Boulder Applied Math
    \includegraphics[height=0.8cm]{./img/appm.png}
}

\frame{%
    \frametitle{High Level Overview}

    \textbf{Can we make a soliton gas?}
    \begin{itemize}
        \item One-Dimensional Fluid System
        \item Solitons (and other discrete structures) possible
        \item Need to use some statistics...
    \end{itemize}
}

\frame{%
    \frametitle{What's our Environment?}
    \begin{columns}[T]
        \column{0.5\textwidth}
        \textbf{Viscous Fluid Conduits}
        \begin{itemize}
            \item Two viscous fluids, with inner forming axisymmetric conduit.
            \item Exterior Fluid: $\rho^{(e)}$ density and $\mu^{(e)}$ viscosity
            \item Interior Fluid: $\rho^{(i)}$ density and $\mu^{(i)}$ viscosity
            \item $\rho^{(i)} < \rho^{(e)} \Rightarrow$ buoyant flow
            \item $\mu^{(i)} << \mu^{(e)} \Rightarrow$ minimal drag
            \item $\text{Re} << 1 \Rightarrow$ low Reynold's number
        \end{itemize}
        \column{0.5\textwidth}
        \begin{figure}[H]
            \centering
            \includegraphics[scale=0.15]{./img/conduit.png}
        \end{figure}
    \end{columns}
}

\frame{%
    \frametitle{Experimental Setup}
    \begin{figure}[H]
        \centering
        \includegraphics[scale=0.13]{./img/setup.png}
    \end{figure}
}

\frame{%
    \frametitle{What's a Soliton Gas?}

    Let's break it down into its components.

    \begin{columns}[T]
        \column{0.5\textwidth}
        \textbf{Soliton}
        \begin{itemize}
            \item ``...self-reinforcing solitary wave packet\ldots (wikipedia)
            \item More generally, a solitary travelling wave.
        \end{itemize}
        \column{0.5\textwidth}
        \textbf{Gas}
    \end{columns}

}

\frame{%
    \frametitle{Numerical Simulations}
}

\frame{%
    \frametitle{Leveraging Parallelism}
}

\frame{%
    \frametitle{Preliminary Results}
}

\frame{%
    \frametitle{Acknowledgements}
    \begin{itemize}
        \item Mark Hoefer
        \item Michelle Maiden
        \item Funded by NSF EXTREEMS-QED
    \end{itemize}
}
\end{document}
