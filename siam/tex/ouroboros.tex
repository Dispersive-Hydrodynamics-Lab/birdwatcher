\documentclass[12pt]{beamer}

\input{./tex/header.tex}
\usepackage{multimedia}
%%%%%%%%%%%%%%%%%%%%%%%%%%%%%%%%%%%%%%%%%%%%%%%%%%%%%%%%%%%%%%%%%%%%%%%%%%%%%%%%
% Beginning of document items - headers, title, toc, etc...
%%%%%%%%%%%%%%%%%%%%%%%%%%%%%%%%%%%%%%%%%%%%%%%%%%%%%%%%%%%%%%%%%%%%%%%%%%%%%%%%
\title[Ouroboros]{Ouroboros}
    \author{Will Farmer\\\url{www.will-farmer.com}}
\institute{University of Colorado, Boulder\\
            Advisors: Mark Hoefer, Michelle Maiden}
\date{March 4, 2017}


%%%%%%%%%%%%%%%%%%%%%%%%%%%%%%%%%%%%%%%%%%%%%%%%%%%%%%%%%%%%%%%%%%%%%%%%%%%%%%%%
% Custom Beamer Theming
%%%%%%%%%%%%%%%%%%%%%%%%%%%%%%%%%%%%%%%%%%%%%%%%%%%%%%%%%%%%%%%%%%%%%%%%%%%%%%%%
\usetheme{Madrid}
\makeatletter

% http://tex.stackexchange.com/questions/35166/how-can-i-remove-the-institute-from-the-author-footline-on-beamer
% TLDR; Remove the Institution from default Madrid theme by redefining footline
% template.
\setbeamertemplate{footline}
{
  \leavevmode%
  \hbox{%
  \begin{beamercolorbox}[wd=.333333\paperwidth,ht=2.25ex,dp=1ex,center]{author in head/foot}%
      \usebeamerfont{author in head/foot}{\color{cugold}\insertshortauthor}%~~\beamer@ifempty{\insertshortinstitute}{}{(\insertshortinstitute)}
  \end{beamercolorbox}%
  \begin{beamercolorbox}[wd=.333333\paperwidth,ht=2.25ex,dp=1ex,center]{title in head/foot}%
    \usebeamerfont{title in head/foot}\insertshorttitle
  \end{beamercolorbox}%
  \begin{beamercolorbox}[wd=.333333\paperwidth,ht=2.25ex,dp=1ex,right]{date in head/foot}%
    \usebeamerfont{date in head/foot}\insertshortdate{}\hspace*{2em}
    \insertframenumber{} / \inserttotalframenumber\hspace*{2ex} 
  \end{beamercolorbox}}%
  \vskip0pt%
}
\makeatother

% Redefine title page to be a little more square/2d
\setbeamertemplate{title page}
{
  \begin{centering}
    \begin{beamercolorbox}[sep=8pt,center]{title}
      \usebeamerfont{title}\inserttitle\par%
      \ifx\insertsubtitle\@empty%
      \else%
        \vskip0.25em%
        {\usebeamerfont{subtitle}\usebeamercolor[fg]{subtitle}\insertsubtitle\par}%
      \fi%     
    \end{beamercolorbox}%
    \begin{beamercolorbox}[sep=8pt,center]{institute}
      \usebeamerfont{institute}\insertinstitute
    \end{beamercolorbox}
    \vskip1em\par
    \begin{beamercolorbox}[sep=8pt,center]{date}
      \usebeamerfont{date}\insertdate
    \end{beamercolorbox}%\vskip0.5em
    \begin{beamercolorbox}[sep=8pt,center]{author}
      \usebeamerfont{author}\insertauthor
    \end{beamercolorbox}
  \end{centering}
  %\vfill
}
\makeatother

\usecolortheme{wolverine}

%%%%% SET THEME
\beamertemplatenavigationsymbolsempty   % Disable navigation

%%%%% INSERT LOGO
\logo{%
    \vspace{-0.3cm}
    \makebox[0.95\paperwidth]{%
        \includegraphics[scale=0.4]{./img/nsf.png}{\color{cublack}Funded by NSF EXTREEMS-QED}
        \hfill%
        \color{cugold}CU Boulder Applied Math\includegraphics[height=0.8cm]{./img/appm.png}
    }%
}

%%%%% DEFINE COLORS
\definecolor{bgcolor}{RGB}{255,255,240}
\definecolor{cugold}{RGB}{207,184,124}
\definecolor{cublack}{RGB}{0,0,0}
\definecolor{cudarkgray}{RGB}{86,90,92}
\definecolor{culightgray}{RGB}{162,164,163}

%%%%% SET COLORS
\setbeamercolor{palette primary}{bg=cugold,fg=cublack}
\setbeamercolor{palette secondary}{bg=culightgray}
\setbeamercolor{palette tertiary}{bg=cublack}
\setbeamercolor{frametitle}{bg=cugold,fg=cublack}
\setbeamercolor{item projected}{fg=cugold,bg=black}
\setbeamercolor{itemize item}{fg=cublack,bg=black}
\setbeamercolor{itemize subitem}{fg=cublack,bg=black}

%%%%% List Styling
\setbeamertemplate{itemize items}[circle]
\setbeamertemplate{enumerate items}[circle]

%%%%%%%%%%%%%%%%%%%%%%%%%%%%%%%%%%%%%%%%%%%%%%%%%%%%%%%%%%%%%%%%%%%%%%%%%%%%%%%%
% Document
%%%%%%%%%%%%%%%%%%%%%%%%%%%%%%%%%%%%%%%%%%%%%%%%%%%%%%%%%%%%%%%%%%%%%%%%%%%%%%%%
\begin{document}
\frame{\titlepage}

\logo{%
    \vspace{-0.3cm}\color{cugold}CU Boulder Applied Math
    \includegraphics[height=0.8cm]{./img/appm.png}
}

\frame{%
    \frametitle{High Level Overview}

    \textbf{Can we make a soliton gas?}
    \begin{itemize}
        \item One-Dimensional Fluid System
        \item Solitons (and other discrete structures) possible
        \item Need to use some statistics...
    \end{itemize}
}

\frame{%
    \frametitle{What's our Environment?}
    \begin{columns}[T]
        \column{0.5\textwidth}
        \textbf{Viscous Fluid Conduits}
        \begin{itemize}
            \item Two viscous fluids, with inner forming axisymmetric conduit.
            \item Exterior Fluid: $\rho^{(e)}$ density and $\mu^{(e)}$ viscosity
            \item Interior Fluid: $\rho^{(i)}$ density and $\mu^{(i)}$ viscosity
            \item $\rho^{(i)} < \rho^{(e)} \Rightarrow$ buoyant flow
            \item $\mu^{(i)} << \mu^{(e)} \Rightarrow$ minimal drag
            \item $\text{Re} << 1 \Rightarrow$ low Reynold's number
        \end{itemize}
        \column{0.5\textwidth}
        \begin{figure}[H]
            \centering
            \includegraphics[scale=0.15]{./img/conduit.png}
        \end{figure}
    \end{columns}
}

\frame{%
    \frametitle{Experimental Setup}
    \begin{figure}[H]
        \centering
        \includegraphics[scale=0.13]{./img/setup.png}
    \end{figure}
}

\frame{%
    \frametitle{Notes on Solitons}
    Our system is governed by the Conduit Equation,
    \begin{align*}
        A_t + \pren{A^2}_z - \pren{A^2 \pren{A^{-1} A_t}_z}_z = 0
    \end{align*}
    \begin{itemize}
        \item Solitons are \textit{solitary travelling waves}.
        \item Solitons are a special solution to the conduit equation of the form
            \begin{align*}
                A(z, t) = f(\zeta) = f(z - ct)
            \end{align*}
        \item Solitons have Non-linear characteristics, most notably their speed
            is determined by their non-dimensionalized amplitude ($a$).
            \begin{align*}
                c = \frac{a^2 - 2a^2 \ln a - 1}{2a -a^2 - 1}
            \end{align*}
    \end{itemize}
}

\frame{%
    \frametitle{Conduit Soliton}

    \begin{figure}[H]
        \centering
        \includegraphics[scale=0.5]{./img/soliton.png}
    \end{figure}
}

\frame{%
    \frametitle{Soliton - Soliton Interactions}

    \begin{itemize}
        \item Two solitons can interact if a bigger one chases a smaller one.
        \item The solitons' speed and amplitude are preserved save for a
            phase-shift
    \end{itemize}

    \begin{center}
        \includegraphics[scale=0.2]{./img/2soli1.png}\\
        \includegraphics[scale=0.2]{./img/2soli2.png}\\
        \includegraphics[scale=0.2]{./img/2soli3.png}
    \end{center}
}

\frame{%
    \frametitle{Phase Shift}

    \includegraphics[scale=0.1]{./img/phaseshift.png}
}

\frame{%
    \frametitle{What's a Soliton Gas?}

    \begin{itemize}
        \item A \textit{Soliton} is a solitary travelling wave, while a
            \textit{Gas} can be thought of as a collection of particles
            interacting.
        \item In our system, our gas is one-dimensional, as we're limited to the
            interior of the conduit.
        \item A soliton gas has inherent random behavior dictated by two random
            variables:
            \begin{enumerate}
                \item Frequency of solitons, $Z$
                \item Soliton amplitude, $A$
            \end{enumerate}
        \item In a ``true'' soliton gas,
            \begin{align*}
                Z_g \sim Poisson(\lambda)
            \end{align*}
            Where $\lambda$ corresponds to the ``density'' of the gas.
    \end{itemize}
}

\frame{%
    \frametitle{How do we find a Soliton Gas?}

    \begin{itemize}
        \item Remember, true gas has $Z_g \sim Poisson(\lambda)$!
        \item Due to the phase-shift from soliton-soliton interactions, any
            input distribution $Z$ will converge to $Z_g$.
        \item The Amplitude distribution $A$ is preserved over time.
        \item Given some initial spacing distribution $Z_0$, we need to measure
            how close it is to our ideal distribution $Z_g$.
    \end{itemize}
}

\frame{%
    \frametitle{Numerical Simulations}

    \begin{itemize}
        \item Experiments are hard and time consuming. Much easier to just run
            simulations.
        \item Can simulate using adaptive step RK4 with periodic boundary
            conditions.
        \item In this system, with periodic boundary conditions, recurrence is
            guaranteed.
    \end{itemize}
}

\frame{%
    \frametitle{Why Periodic Boundary Conditions?}

    \begin{itemize}
        \item Recurrence is guaranteed.
        \item Therefore we'll run two simulations simultaneously, one on $[0,L]$
            and the other on $[0,2L]$.
        \item At each timestep we'll check ``poissonness'' of each. If they
            differ by more than 20\%, we restart with new initial conditions.
        \item Since we have recurrence, eventually the simulation on $[0,L]$
            will tend back to initial conditions. We want to stop before then.
        \item Similarly, due to phase shifts, poissonness should increase and
            then decrease. We want to stop before this.
    \end{itemize}
}

\frame{%
    \frametitle{What are we calling ``Poissonness''?}

    \begin{itemize}
        \item We've established that a soliton gas should have
            Poisson-distributed solitons.
        \item This means that we can look at the problem as a Poisson-Point
            Process, i.e. at any given point in space we should see points
            appear over time.
            \begin{figure}[H]
                \centering
                \includegraphics[scale=0.5]{./img/HPP.png}
            \end{figure}
    \end{itemize}
}

\frame{%
    \frametitle{Meaning....}

    \begin{itemize}
        \item Since this is a Poisson-Point Process the gap between points is
            exponentially distributed.
        \item Therefore the ``Poissonness'' is how close our gaps of our
            solitons are to the exponential distribution.
        \item We use the residual sum squared on the QQ-plot as a metric of
            ``distance'' from one distribution to the other. This is our
            Poissonness metric.
    \end{itemize}
}

\frame{%
    \frametitle{Initial Conditions}

    \begin{itemize}
        \item We have two random variables to simulate.
        \item Very first case is easy
        \begin{itemize}
            \item $Z$ is one soliton per ``unit''
            \item $A$ is $Unif\pren{\cren{2,2.5,3,3.5,4,4.5,5,5.5,6,6.5,7}}$
        \end{itemize}
    \item After a restart on $2L$ and $4L$ things are trickier... We can't feed
        initial conditions where solitons overlap (makes solver grumpy) so
        instead we need to simulate initial conditions with same Poissonness as
        last timestep (before we quit)
    \end{itemize}
}

\frame{%
    \frametitle{Plot of Initial Conditions}

    \begin{figure}[H]
        \centering
        \includegraphics[scale=0.28]{./img/IC.png}
    \end{figure}
}

\frame{%
    \frametitle{Leveraging Parallelism}

    The big flaw so far is that we're only looking at a single run of the
    simulation. We could easily get bad results from only a single run.

    \vspace{1cm}
    Let's instead consider running a hundred different simulations
    simultaneously, or even a thousand. We have to adjust our simulation to be
    able to handle running in a massively parallel environment such as the CU
    supercomputer, Summit.
}

\frame{%
    \frametitle{Leveraging Parallelism}

    \begin{figure}[H]
        \centering
        \includegraphics[width=\textwidth]{./img/threads.png}
    \end{figure}
}

\frame{%
    \frametitle{Preliminary Results}

    \begin{figure}[H]
        \centering
    \end{figure}
}

\frame{%
    \frametitle{References and Acknowledgements}
    \textbf{References}
    \begin{itemize}
        \item D. S. Agafontsev and V. E. Zakharov, Nonlinearity 28, 2791 (2015).
    \end{itemize}
    \textbf{Acknowledgements}
    \begin{itemize}
        \item Mark Hoefer
        \item Michelle Maiden
        \item Funded by NSF EXTREEMS-QED
    \end{itemize}

}
\end{document}
